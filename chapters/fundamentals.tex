%%%%%%%%%%%%%%%%%%%%%%%%%%%%%%%%%%%%%%%%%%%%%%%%%%%%%%%%%%%%%%%%%%%%%%%%
\chapter{Theoretical Background}
\label{sec:fundamentals}
%%%%%%%%%%%%%%%%%%%%%%%%%%%%%%%%%%%%%%%%%%%%%%%%%%%%%%%%%%%%%%%%%%%%%%%%

A \emph{sponge function} or a \emph{sponge construction} is a class of algorithms with finite internal state that take an input bit stream of any length and produce an output bit stream of any desired length \cite{wiki:2017:sponge-function}. The internal state \(S\) consists of \(w = b + c\) bits, where \(w\) is known as the \emph{width} of the sponge, \(b\) is called its \emph{bitrate} and \(c\) its \emph{capacity}. At the heart of every sponge function lies a \emph{fixed-width permutation} \(f\) which takes \(w\) bits as input and produces \(w\) bits as output. A common example of \(f\) is Blake2b, which in its 64-bit modification operates on 128-byte chunks.

As shown in figure \ref{figure:sponge-construction}, a sponge function conceptually consists of two parts: the absorbing and the squeezing phase. To start with, an incoming message is padded so that its length is a multiple of \(b\) bits. Secondly, the internal state of the sponge is initialized with zeros. Then the absorbing phase begins: the next \(b\) bits of the message are XOR-ed with the first \(b\) bits of \(S\) and then the \(f\) permutation is applied to the whole \(S\). This process repeats until the entire message has been absorbed. At this point the squeezing phase starts: the first \(b\) bits of \(S\) are emitted and then the \(f\) permutation is applied to the whole \(S\). This keeps going until the desired output length is reached.

\begin{figure}
  \begin{tikzpicture}[
  ostyle/.style={
      shape=rectangle
      ,draw
      ,minimum width=0.5cm,
    }
    ,fstyle/.style={
      shape=rectangle
      ,draw
      ,rounded corners=5pt
      ,minimum width=0.5cm
      ,minimum height=5cm
      }
    ,xstyle/.style={
    }
  ]

  \coordinate (A0) at (-0.5, 5) {};
  \coordinate (A1) at (-0.5, 3) {};
  \coordinate (A2) at (-0.5, 0) {};

  \draw [<->,semithick] (A0) to node [left] {\(b\)} (A1);
  \draw [<->,semithick] (A1) to node [left] {\(c\)} (A2);

  \node [ostyle,minimum height=2cm] (00) at (0.5, 4  ) {\(0\)};
  \node [ostyle,minimum height=3cm] (01) at (0.5, 1.5) {\(0\)};

  \node [shape=rectangle,draw] (padded message) at (-0.5, 6) {padded message};

  \coordinate (y0) at (1.5, 6) {};
  \coordinate (y1) at (3.5, 6) {};
  \coordinate (y2) at (5.5, 6) {};
  \coordinate (y3) at (7.5, 6) {};
  \coordinate (y4) at (9.5, 6) {};
  \draw [-] (padded message) to (y0) to (y1) to (y2);

  \node [xstyle] (x0) at (1.5, 4) {\(\oplus\)};
  \node [xstyle] (x1) at (3.5, 4) {\(\oplus\)};
  \node [xstyle] (x2) at (5.5, 4) {\(\oplus\)};
  \coordinate (x3) at (7.5, 4) {};
  \coordinate (x4) at (9.5, 4) {};

  \node [shape=rectangle] (absorb) at (3.5, -1) {absorbing phase};
  \node [shape=rectangle] (squeeze) at (9.5, -1) {squeezing phase};

  \draw [thick,dashed] (7.25, 6.5) -- (7.25, -1.5);

  \draw [->] (y0) to node [right,near start] {1\textsuperscript{st} \(b\) bits} (x0);
  \draw [->] (y1) to node [right,near start] {2\textsuperscript{nd} \(b\) bits} (x1);
  \draw [->] (y2) to node [right,near start] {3\textsuperscript{rd} \(b\) bits} (x2);
  \draw [->] (x3) to node [right,near end] {1\textsuperscript{st} \(b\) bits} (y3);
  \draw [->] (x4) to node [right,near end] {2\textsuperscript{nd} \(b\) bits} (y4);

  \node [fstyle] (f0) at ( 2.5, 2.5) {\(f\)};
  \node [fstyle] (f1) at ( 4.5, 2.5) {\(f\)};
  \node [fstyle] (f2) at ( 6.5, 2.5) {\(f\)};
  \node [fstyle] (f3) at ( 8.5, 2.5) {\(f\)};
  \node [fstyle] (f4) at (10.5, 2.5) {\(f\)};

  \draw [->] (00) to (x0);
  \draw [->] (x0) -- (x0 -| f0.west);
  \draw [->] ($(f0.east) + (0, 1.5)$) -- (x1);
  \draw [->] (x1) -- (x1 -| f1.west);
  \draw [->] ($(f1.east) + (0, 1.5)$) -- (x2);
  \draw [->] (x2) -- (x2 -| f2.west);
  \draw [-] ($(f2.east) + (0, 1.5)$) -- (x3);
  \draw [-] (x3) -- ($(f3.west) + (0, 1.5)$);
  \draw [-] ($(f3.east) + (0, 1.5)$) -- (x4);
  \draw [-] (x4) -- ($(f4.west) + (0, 1.5)$);
  \draw [->] ($(f4.east) + (0, 1.5)$) -- ($(f4.east) + (1, 1.5)$);

  \draw [->] (01) -- (01 -| f0.west);
  \draw [->] ($(f0.east) + (0, -1)$) -- ($(f1.west) + (0, -1)$);
  \draw [->] ($(f1.east) + (0, -1)$) -- ($(f2.west) + (0, -1)$);
  \draw [->] ($(f2.east) + (0, -1)$) -- ($(f3.west) + (0, -1)$);
  \draw [->] ($(f3.east) + (0, -1)$) -- ($(f4.west) + (0, -1)$);
  \draw [->] ($(f4.east) + (0, -1)$) -- ($(f4.east) + (1, -1)$);

  \end{tikzpicture}
  \caption{The Sponge Construction}
  \label{figure:sponge-construction}
\end{figure}

\begin{figure}
  \begin{tikzpicture}[
  ostyle/.style={
      shape=rectangle
      ,draw
      ,minimum width=0.5cm,
    }
    ,fstyle/.style={
      shape=rectangle
      ,draw
      ,rounded corners=5pt
      ,minimum width=0.5cm
      ,minimum height=5cm
      }
    ,xstyle/.style={
    }
  ]

  \coordinate (A0) at (-0.5, 5) {};
  \coordinate (A1) at (-0.5, 3) {};
  \coordinate (A2) at (-0.5, 0) {};

  \draw [<->,semithick] (A0) to node [left] {\(b\)} (A1);
  \draw [<->,semithick] (A1) to node [left] {\(c\)} (A2);

  \node [ostyle,minimum height=2cm] (00) at (0.5, 4  ) {\(0\)};
  \node [ostyle,minimum height=3cm] (01) at (0.5, 1.5) {\(0\)};

  \coordinate (x00) at ( 1.5, 6) {};
  \coordinate (x01) at ( 4.5, 6) {};
  \coordinate (x02) at ( 7.5, 6) {};
  \coordinate (x03) at (10.5, 6) {};

  \node [xstyle] (x10) at ( 1.5, 4) {\(\oplus\)};
  \node [xstyle] (x11) at ( 4.5, 4) {\(\oplus\)};
  \node [xstyle] (x12) at ( 7.5, 4) {\(\oplus\)};
  \node [xstyle] (x13) at (10.5, 4) {\(\oplus\)};

  \draw [->] (x00) to node [anchor=south,rotate=90] {1\textsuperscript{st} \(b\) bits} (x10);
  \draw [->] (x01) to node [anchor=south,rotate=90] {2\textsuperscript{nd} \(b\) bits} (x11);
  \draw [->] (x02) to node [anchor=south,rotate=90] {3\textsuperscript{rd} \(b\) bits} (x12);
  \draw [->] (x03) to node [anchor=south,rotate=90] {4\textsuperscript{th} \(b\) bits} (x13);

  \coordinate (x14) at (13.5, 4) {};
  \coordinate (x24) at (13.5, 1.5) {};

  \coordinate (y00) at (3.5, 6) {};
  \coordinate (y01) at (6.5, 6) {};
  \coordinate (y02) at (9.5, 6) {};

  \coordinate (y10) at (3.5, 4) {};
  \coordinate (y11) at (6.5, 4) {};
  \coordinate (y12) at (9.5, 4) {};

  \draw [<-] (y00) to node [anchor=south,rotate=90] {1\textsuperscript{st} \(b\) bits} (y10);
  \draw [<-] (y01) to node [anchor=south,rotate=90] {2\textsuperscript{nd} \(b\) bits} (y11);
  \draw [<-] (y02) to node [anchor=south,rotate=90] {3\textsuperscript{rd} \(b\) bits} (y12);

  \draw [->] (y10) -- (y00);
  \draw [->] (y11) -- (y01);
  \draw [->] (y12) -- (y02);

  \node [fstyle] (f0) at ( 2.5, 2.5) {\(f\)};
  \node [fstyle] (f1) at ( 5.5, 2.5) {\(f\)};
  \node [fstyle] (f2) at ( 8.5, 2.5) {\(f\)};
  \node [fstyle,dashed] (f3) at (11.5, 2.5) {\(f\)};

  \draw [->] (00) to (x10);
  \draw [->] (x0) -- (x0 -| f0.west);
  \draw [->] ($(f0.east) + (0, 1.5)$) -- (x11);
  \draw [->] (y10) -- (y00);
  \draw [->] (x11) -- (x11 -| f1.west);
  \draw [->] ($(f1.east) + (0, 1.5)$) -- (x12);
  \draw [->] (x12) -- (x12 -| f2.west);
  \draw [->] ($(f2.east) + (0, 1.5)$) -- (x13);
  \draw [->] (x13) -- ($(f3.west) + (0, 1.5)$);
  \draw [->,dashed] ($(f3.east) + (0, 1.5)$) -- (x14);
  % \draw [-] ($(f3.east) + (0, 1.5)$) -- (x4);
  % \draw [-] (x4) -- ($(f4.west) + (0, 1.5)$);
  % \draw [->] ($(f4.east) + (0, 1.5)$) -- ($(f4.east) + (1, 1.5)$);

  \draw [->] (01) -- (01 -| f0.west);
  \draw [->] ($(f0.east) + (0, -1)$) -- ($(f1.west) + (0, -1)$);
  \draw [->] ($(f1.east) + (0, -1)$) -- ($(f2.west) + (0, -1)$);
  \draw [->] ($(f2.east) + (0, -1)$) -- ($(f3.west) + (0, -1)$);
  \draw [->,dashed] ($(f3.east) + (0, -1)$) -- (x24);
  % \draw [->] ($(f3.east) + (0, -1)$) -- ($(f4.west) + (0, -1)$);
  % \draw [->] ($(f4.east) + (0, -1)$) -- ($(f4.east) + (1, -1)$);

  \end{tikzpicture}
  \caption{The Duplex Construction}
  \label{figure:duplex-construction}
\end{figure}

%=======================================================================
\section{State of the Art}
%=======================================================================
