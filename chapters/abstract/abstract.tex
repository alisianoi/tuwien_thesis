\cleardoublepage
\phantomsection\pdfbookmark[1]{Abstract}{sec:Abstract}

\begin{otherlanguage}{english}
\chapter*{Abstract}

This work describes the process of porting a password hashing scheme called Lyra2 from C99 to Java 1.8. A native Java implementation should ease integration with other Java projects and simplify its usage on other platforms, such as Android phones.

This paper follows a straightforward approach. First, a brief description of the basic Lyra2 configuration is given, followed by an overview of the reference implementation. Then follows a description of the ported Java project along with a number of notable challenges.

Section \ref{sec:results} presents the results. It opens with both manual and automated testing which shows that both projects generate the same hash values when given the same inputs. A performance comparison follows in section \ref{sec:performance-comparison}, showing several thousands of measurements. Finally, section \ref{sec:mobile-application} demonstrates a small proof-of-concept mobile application that uses Lyra2.

In summary, the ported Java project could be found on GitHub \cite{github:2017:lyra2-java} and on Maven Central \cite{maven:2017:lyra2}. The comparison project and the source code for the mobile application are hosted on GitHub as well, projects \cite{github:2017:lyra2-compare} and \cite{github:2017:lyra2-mobile} respectively. Generally, the ported implementation requires more processing time and memory. The main reasons for that could be found in comparison conclusion \ref{sec:comparison-conclusion}. However, it integrates easily into the Android ecosystem which may justify the performance drop.

\bigskip

\section*{Keywords}
Password Hashing Scheme, Lyra2, Memory Hard Functions, Java, Android

\end{otherlanguage}
