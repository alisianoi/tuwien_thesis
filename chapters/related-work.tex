\chapter{Related Work}
\label{chapter:related-work}

\section{Cryptographic Competitions}

It is common practice to announce a competition in order to develop standard cryptographic primitives. For example, the symmetric block cipher Rijndael \cite{daemen:2002:DRA} was chosen to become the Advanced Encryption Standard (AES) \cite{aes-fips} in a competitive selection process that lasted from 1997 to 2000. The competition was organized by the National Institute of Standards and Technology (NIST) and included 15 different designs which were narrowed down to 5 during the final phase: Rijndael, Serpent, Twofish, RC6 and MARS. Such an open standartization process was highly praised by the cryptographic community. So, another competition was held by NIST between 2007 and 2012 in order to select the next hash function standard, SHA-3. The final round included 5 designs: BLAKE, Grøstl, JH, Sklein and Keccak, the last of which ultimately became the winner of the competition.

Choosing cryptographic primitives through an open competitive process is an effective approach. Therefore, when the need for an updated, memory-hard password hashing scheme became apparent, a Password Hashing Competition was held. This time, however, it was not organized by NIST but directly by the cryptographic community. In particular, well-known cryptographers Dmitry Khovratovich and Jean-Philippe Aumasson are listed as direct contacts on the password-hashing.net website. The password hashing competition concluded in 2015, with Argon2 selected as its winner and Catena, Lyra2, Makwa and yescrypt receiving special recognition.

Lyra2 is the primary focus of this work and will be discussed later. This section is meant to provide an overview of other related password hashing schemes.

\section{Main Features of Catena}

\textsc{Catena} is a password-scrambling framework based on bit-reversal graphs. It packs such features as \emph{client-independent updates} which allows a hash value to be updated using larger time or memory cost parameters without having to wait for a user to re-login. \textsc{Catena} also offers a \emph{server-relief} feature which allows to offload most of the hash computation to the client machine rather than the server, hence increasing the number of concurrent logins.

The \textsc{Catena-Butterfly(-Full)} and \textsc{Catena-Dragonfly(-Full)} are the most notable configurations of \textsc{Catena}. The former one is recommended when memory-hardness is important. The \textsc{-Full} versions utilize the complete set of rounds of the underlying hash function Blake2b.

\section{Main Features of Makwa}

The core of Makwa is a "sequence of squarings of a composite (Blum) integer" \cite{pornin:2015:makwa}. A Blum integer is a natural integer \(n\) which can be represented as a \(pq\) where \(p\) and \(q\) are prime numbers with an additional property of

\begin{IEEEeqnarray}{rCl}
    p &=& 3 \texttt{ mod } 4 \\
    q &=& 3 \texttt{ mod } 4
\end{IEEEeqnarray}

The squaring is primarily computation intensive and does not require a significant amount of memory. The \(p\) and \(q\) integers should be kept secret, if they are known then the computation can be accelerated considerably.

Makwa supports \emph{delegation} which is described as the ability to offload part of the computation to an untrusted third party. As well as that, \emph{client-independent updates} are also supported.

\section{Main Features of yescrypt}

\emph{yescrypt} \cite{peslyak:2015:yescrypt} improves upon its predecessor, \emph{scrypt} \cite{percival:2016:scrypt}. However, yescrypt author Alexander Peslyak makes it clear that the author of scrypt is a different person, namely Colin Percival. The yescrypt PHS clarifies some minor inconsistencies discovered with its predecessor and provides a compatibility mode. It also introduces a novel configuration with a read-only memory (ROM) table. In that configurations random lookups are performed so as to ensure that this table must remain in memory. Finally, the \texttt{YESCRYPT\_RW} flag enables these lookups as well as a number of optimized instructions.

\section{Main Features of Argon2}

\emph{Argon2} has two distinct configurations: \emph{Argon2i} and \emph{Argon2d}. The former revists the blocks of the in-memory matrix in the data-\emph{independent} fashion while the latter does so in a data-\emph{dependent} manner. This means that Argon2i is better suited for scenarios when \emph{side-channel attacks} are a viable concern, such as during password hashing or key derivation. At the same time Argon2d is more resistant to \emph{time-memory tradeoffs} which makes it more suitable for digital cryptocurrencies or other cases where proof-of-work is important.

Argon2 accepts the following set of parameters: password, salt, degree of paralelism, length of the produced hash (called a \emph{tag} in \cite{biryukov:2015:argon2}), memory cost, time cost, version number (for compatibility reasons, currently at \texttt{0x13}), secret value, associated data and type of configuration to use (Argon2i or Argon2d).

The more interesting parameters are the degree of parallelism as well as secret value together with associated data. The last of these three adds more flexibility to the scheme. The secret value parameter enables \emph{keyed-hashing} and improves security in case of a database leak. The degree of parallelism directly corresponds to the number of rows of the in-memory matrix.
