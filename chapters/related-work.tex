\chapter{Related Work}
\label{chapter:related-work}

\section{Cryptographic Competitions}

It is common practice to announce a competition in order to develop standard cryptographic primitives. For example, the symmetric block cipher Rijndael was chosen to become the Advanced Encryption Standard (AES) in a competitive selection process that lasted from 1997 to 2000. The competition was organized by the National Institute of Standards and Technology (NIST) and included 15 different designs which were narrowed down to 5 during the final phase: Rijndael, Serpent, Twofish, RC6 and MARS. Such an open standartisation process was highly praised by the cryptographic community. So, another competition was held by NIST between 2007 and 2012 in order to select the next hash function standard, SHA-3. The final phase again included 5 designs: BLAKE, Grøstl, JH, Sklein and Keccak, the last of which ultimately became the winner of this second competition.

Choosing cryptographic primitives through an open competitive process is an effective approach. Therefore, it should not come as a surprise that when the need for an updated, memory-hard password hashing scheme became apparent, a Password Hashing Competition was held. This time, however, it was not organized by NIST but directly by the cryptographic community. In particular, well-known cryptographers Dmitry Khovratovich and Jean-Philippe Aumasson are listed as direct contacts on the password-hashing.net website. The password hashing competition concluded in 2015, with Argon2 selected as its winner and Catena, Lyra2, Makwa and yescrypt receiving special recognition. 
