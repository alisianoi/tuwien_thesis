%%%%%%%%%%%%%%%%%%%%%%%%%%%%%%%%%%%%%%%%%%%%%%%%%%%%%%%%%%%%%%%%%%%%%%%%
\chapter{Introduction}
\label{sec:introduction}
%%%%%%%%%%%%%%%%%%%%%%%%%%%%%%%%%%%%%%%%%%%%%%%%%%%%%%%%%%%%%%%%%%%%%%%%

\makeatletter
\makeatother

In der Einleitung soll die Zielsetzung der Arbeit beschrieben, ihre Einordnung in einen übergeordneten Kontext hergestellt und die Bedeutung des Themas erörtert werden.
\begin{itemize}
	\item Problemstellung
	\item Motivation
	\item Zielsetzung
	\item Aufbau der Arbeit
\end{itemize}

%=======================================================================
\section{Problem setting}
%=======================================================================

Formulierung der Problemstellung, Einbettung in das Forschungsumfeld und Theorie, auf die sich die Arbeit beziehen. Tendenziell kurz, allgemeiner und sehr gut verständlich -- detaillierter im Kapitel \enquote{Grundlagen}.

%=======================================================================
\section{Motivation}
%=======================================================================

In diesem Kapitel wird der Forschungsbedarf aufgezeigt. Nach dem Lesen dieses Kapitels sollten folgende Punkte klar dargestellt sein:
\begin{itemize}
	\item Aktueller Stand der Wissenschaft in Bezug auf die zuvor formulierte Problemstellung und klare Darstellung, was hier unzureichend/offen ist.
	\item GGf. Darstellung des größeren Forschungsbereichs, in den die Diplomarbeit eingebettet ist.
	\item Darlegung der Bedeutung des Themas für den Stand oder die Weiterentwicklung eines Bereichs der Informatik (z.B. Datenbanksysteme, Mobile Anwendungen, Java-Programmierung, Rechenzentrumsbetrieb, \dots) oder eines Fachbereichs (z.B. Bankwesen, Wertpapierhandel, Gesundheitswesen, Transportwesen, Flugsicherheit \dots). Erklärung, was durch die Lösung des Problems z.B. kostengünstiger/schneller/hochwertiger/sicherer/anwendbarer/\enquote{schöner} etc. wird.
\end{itemize}

%=======================================================================
\section{Methodological approach}
%=======================================================================

Nachdem die Problemstellung und die Motivation zu deren Lösung formuliert wurden, wird in diesem Kapitel das zu erarbeitende Resultat beschrieben.

Nach dem Lesen dieses Kapitels sollten folgende Punkte klar dargestellt sein:
\begin{itemize}
	\item Umfang, in dem die Problemstellung am Ende der Arbeit gelöst sein soll bzw. mit welchen Einschränkungen.
	\item Methode zur Erarbeitung des Resultats.
	\item Gibt es einen Theorie- und einen Praxisteil?
	\item Schwerpunkte des Praxisteils (z.B. Durchführung einer Umfrage, Programmierung, Herstellung von Hardware, Erprobung einer Methode in einem konkreten Projekt)?
	\item Art des Resultats (z.B. ein Programm, eine Formel, eine Methode, die Erweiterung einer existierenden Methode, ein Konzept, ein Framework, Hardware-Prototyp, eine bewiesene Erkenntnis)?
\end{itemize}

%=======================================================================
\section{Structure of this work}
%=======================================================================

% \begin{itemize}
%   \item Cover page
%   \item Acknowledgements
%   \item Abstract of the thesis in English and German
%   \item Table of contents
%   \item Introduction
%   	\begin{itemize}
%   		\item motivation
%   		\item problem statement (which problem should be solved?)
%   		\item aim of the work
%   		\item methodological approach
%   		\item structure of the work
%   	\end{itemize}
%   \item State of the art / analysis of existing approaches
%   	\begin{itemize}
%   		\item literature studies
%   		\item analysis
%   		\item comparison and summary of existing approaches
%   	\end{itemize}
%   \item Methodology
%   	\begin{itemize}
%   		\item used concepts
%   		\item methods and/or models
%   		\item languages
%   		\item design methods
%   		\item data models
%   		\item analysis methods
%   		\item formalisms
%   	\end{itemize}
%   \item Suggested solution/implementation
%   \item Critical reflection
%   	\begin{itemize}
%   		\item comparison with related work
%   		\item discussion of open issues
%   	\end{itemize}
%   \item Summary and future work
%   \item Appendix: source code, data models, \dots
%   \item Bibliography
% \end{itemize}
%
