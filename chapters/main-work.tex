%%%%%%%%%%%%%%%%%%%%%%%%%%%%%%%%%%%%%%%%%%%%%%%%%%%%%%%%%%%%%%%%%%%%%%%%
\chapter{Details of the Ported Implementation}
\label{sec:problemdescription}
%%%%%%%%%%%%%%%%%%%%%%%%%%%%%%%%%%%%%%%%%%%%%%%%%%%%%%%%%%%%%%%%%%%%%%%%

\begin{itemize}
	\item Im Kapitel \enquote{Design} sollte die konzeptionelle Lösung vorgestellt, diskutiert und begründet werden. Das Ergebnis dieses Kapitels könnte beispielsweise eine Protokoll-Architektur sein.
	\item Im Kapitel \enquote{Modelle} erfolgt üblicherweise das Feindesign. In diesem Kapitel könnten beispielsweise einzelne Protokolle bzw. Algorithmen aus der vorher definierten Protokoll-Architektur eingeführt und diskutiert werden. Achtung: Generell darauf achten, bei der eingangs erläuterten Notation zu bleiben und nicht Synonyme zu verwenden, verwirrt den Leser.
	\item Das Kapitel \enquote{Implementierung} sollte sich dann vorwiegend mit den Details der Umsetzung befassen. In diesem Kapitel sollte nur im Ausnahmefall exemplarisch Quellcode vorgesehen werden. Vielmehr sollten alle Probleme, die bei der Realisierung aufgetreten sind, dokumentiert, interpretiert und die Lösung erläutert werden.
\end{itemize}
