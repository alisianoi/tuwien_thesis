%%%%%%%%%%%%%%%%%%%%%%%%%%%%%%%%%%%%%%%%%%%%%%%%%%%%%%%%%%%%%%%%%%%%%%%%
\chapter{Conclusion}
\label{sec:conclusion}
%%%%%%%%%%%%%%%%%%%%%%%%%%%%%%%%%%%%%%%%%%%%%%%%%%%%%%%%%%%%%%%%%%%%%%%%

The reference implementation of Lyra2 was successfully ported from C99 to Java 1.8. The ported counterpart generally works slower and requires more memory. This can be explained by the numerous extra steps that must take place in order to ensure algorithm-level compatibility (i.e. producing the same hash values when given the same inputs). There are scenarios when performance can be traded for ease of integration and adoption, such as the case with smartphones. In the Android ecosystem the ported Lyra2 implementation can be integrated as a matter of a few clicks of the mouse button.

In my opinion, there are several viable options for further research. Firstly, algorithm-level compatibility can be sacrificed in favour of performance. However, personally I believe that producing the same hash values is key for successful adoption. Secondly, the parallel version of Lyra2 could be ported into Java. This would introduce another interesting layer for compatibility questions: concurrency. Finally, an exhaustive optimization effort of the Java implementation could be attempted as well. Although I have successfully found one significant optimization trick, there are likely to be more.
